\documentclass[UTF8,a4paper,11pt]{ctexart}
\usepackage{listings} 
\usepackage{xcolor} 
\usepackage{amsmath}
\lstset{
    basicstyle=\tt,
    keywordstyle=\color{purple}\bfseries,
    identifierstyle=\color{brown!80!black},
    commentstyle=\color{gray}
    showstringspaces=false,
    numbers = left,                
    numberstyle = \small,               
}
\title{高等社线代第五章笔记}
\author{5eqn}
\date{\today}
\begin{document}
  \maketitle
  \section{特征向量和特征值}
    找到一个一维的 Invariant Subspace,
    子空间中的放缩倍率是特征值 $\lambda$,
    空间中任意一个非零向量都是特征向量. 
  \section{特征方程}
    构造一个方程, 使得其每个解都是一个特征值:
    \[
    \begin{aligned}
      \mathrm{det}(\lambda I-A)=0
    .\end{aligned}
    \] 
    变形后可以得到
    \[
    \begin{aligned}
      \prod_{i=1}^{n} \lambda_i=\mathrm{det}(A)
    .\end{aligned}
    \] 
  \section{相似对角化}
    定义相似关系为
    \[
    \begin{aligned}
      P^{-1}AP=B \iff A \sim B
    .\end{aligned}
    \] 
    那么如果 $A$ 是特征向量已知的矩阵, 
    $B$ 是对角矩阵, 
    $P$ 将基向量映射到$A$ 的特征向量, 
    由于
    \[
    \begin{aligned}
      AP=PB
    .\end{aligned}
    \] 
    容易看出可以通过这种方式, 
    由 $A$ 构造出相似对角矩阵$B$.

    注意, $A$ 进行 Polar Decomposition 之后
    得到的 Isometry 不能含有旋转元素,
    这可以通过对矩阵减掉重根特征值之后算 Rank 来检验.
    如果含有旋转元素, 会至少多出两个 Rank.
    例如, 对于矩阵
    \[
    \begin{aligned}
      \begin{pmatrix} 
        \cos \theta & -\sin \theta & 0\\
        \sin \theta & \cos \theta & 0 \\
        0 & 0 & 1
      \end{pmatrix} 
    .\end{aligned}
    \] 
    容易发现 1 是一个特征值,
    假设 1 有重根,
    将矩阵减去单位阵之后, 
    假如 $A$ 可以相似对角化,
    应当得到 Rank 为 0 的矩阵.
    实际情况是得到了 Rank 为 2 的矩阵,
    因此 $A$ 不能相似对角化.
  \section{谱定理}
    对于复正交矩阵或实对称矩阵,
    存在由特征向量组成的规范正交基,
    因此不同特征值对应的特征向量也相互平行.
    对于复数的情况, 我们首先需要证明正交矩阵保持模:
    \[
    \begin{aligned}
      \langle T^T T v, v \rangle = \langle T T^T v, v \rangle
      \implies |Tv| = |T^Tv|
    .\end{aligned}
    \] 
    容易知道 $T$ 可以变成上三角矩阵,
    并且 $T$ 如果是上三角矩阵, 那么它只能是对角阵.

    对于实对称矩阵, 容易发现不可分解的 $T^2+bT+c$ 可逆,
    但只通过不断应用可分解或不可分解的二次式,
    在其可以分解为 $v, Tv, T^2v, \ldots, T^n v$ 的形式后,
    最终的结果一定不可逆.
    因此, 不可逆的二次式一定可分解成这种形式:
    \[
    \begin{aligned}
      (T-\lambda_1 I)(T-\lambda_2 I)
    .\end{aligned}
    \] 
    这样 $T$ 必定有特征值.

    同时, 容易发现 $T$ 去掉不变子空间之后,
    依然能保持对称性,
    因此可以通过归纳推出需要的结论.

    如果只想推出不同特征值对应的特征向量正交,
    考虑采用以下思路,
    虽然我并不清楚这个推导路径
    是怎么从指数复杂度的公式变形无向图中搜索到的:
    \[
    \begin{aligned}
      A \alpha_1=\lambda_1 \alpha_1
      &\implies \alpha_1^T A=\lambda_1 \alpha_1^T\\
      &\implies \alpha_1^T A \alpha_2=\lambda_1 \alpha_1^T \alpha_2\\
      &\implies \lambda_2 \alpha_1^T \alpha_2 = \lambda_1 \alpha_1^T \alpha_2\\
      &\implies \langle \alpha_1, \alpha_2 \rangle = 0
    .\end{aligned}
    \] 
\end{document}
